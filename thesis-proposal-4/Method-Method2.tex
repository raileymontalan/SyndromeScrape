\section{Support Vector Machines}
Support Vector Machines (SVMs) are hyperplanes that divide a given dataset into two sections. All vectors to the left of the hyperplane are denoted as -1, while those on the right are 1. The vectors that lie closest to the hyperplane are known as the support vectors. Moreover, the hyperplane divides the training set in such a way that the margin between the hyperplane and the training vectors is maximized \cite{tong2001support}.

One notable feature of SVMs is the kernel. The kernel function is used to project the given dataset into a higher dimension when necessary, making it possible to find a hyperplane to divide linearly-inseparable data sets \cite{tong2001support}, making the problem a quadratic optimization problem with one global solution. This gives the algorithm a computational advantage not present in some other classification methods \cite{ben2001support}.

SVMs will be used to classify the articles into health-related and non-health-related. Only the health-related articles will be included in the final database of information.
